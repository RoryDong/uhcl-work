\documentclass{article}
\usepackage{fullpage}
\usepackage{amsmath, amsfonts, amscd, epsfig, amssymb, amsthm}
\usepackage[]{mcode}
\usepackage{listings}
\usepackage{sectsty}
\usepackage{color} %red, green, blue, yellow, cyan, magenta, black, white
\definecolor{mygreen}{RGB}{28,172,0} % color values Red, Green, Blue
\definecolor{mylilas}{RGB}{170,55,241}
\sectionfont{\centering}
\subsectionfont{}

\begin{document}

\lstset{language=Matlab,%
    %basicstyle=\color{red},
    breaklines=true,%
    morekeywords={matlab2tikz},
    keywordstyle=\color{blue},%
    morekeywords=[2]{1}, keywordstyle=[2]{\color{black}},
    identifierstyle=\color{black},%
    stringstyle=\color{mylilas},
    commentstyle=\color{mygreen},%
    showstringspaces=false,%without this there will be a symbol in the places where there is a space
    numbers=left,%
    numberstyle={\tiny \color{black}},% size of the numbers
    numbersep=9pt, % this defines how far the numbers are from the text
    emph=[1]{for,end,break},emphstyle=[1]\color{red}, %some words to emphasise
    %emph=[2]{word1,word2}, emphstyle=[2]{style},    
}


\title{CENG5131 Homework Set \#2}
\author{Mike Moore}
\maketitle

%\setcontext{1}{1}

\newpage

%----------------------------------------------------------------------------------------
%  Problem 1
%----------------------------------------------------------------------------------------

%%%%%%%%%%%%%%%%%%%%%%%%%%%%%%%%%%%%%%%%%%%%%%%%%%%%%%%%%%%%
\section*{Problem 1}
\subsection*{Problem Statement}

Find the locus of points in the complex plane that satisify the following
equation:


\[ \frac{z}{z-1} = 2, \]

\subsection*{Problem Solution}
Solving this problem will require substituing $z$ for $x + iy$ and 


\newpage

%----------------------------------------------------------------------------------------
%  Problem 2
%----------------------------------------------------------------------------------------

%%%%%%%%%%%%%%%%%%%%%%%%%%%%%%%%%%%%%%%%%%%%%%%%%%%%%%%%%%%%
\section*{Problem 2}
\subsection*{Problem Statement}

Find the solutions to the equation $ z^3 = - 1 $ and write the answers as both 
$a + ib$ and $|z|\angle\theta$. Then compute the cube root using MATLAB's ``roots'' command and multiply them together
to check.

\subsection*{Problem Solution}
Solving this problem will require using De Moivre's theorem: 
\[ a = r(\cos\theta + i\sin N\theta) \]
Using this theorem we can write the roots of a as:
\[ a^{1/N} = \sqrt[\leftroot{-2}\uproot{2}N]{r}[(\cos{(\frac{\theta}{N} + p\frac{2\pi}{N})} + i\sin{(\frac{\theta}{N} + p\frac{2\pi}{N})}] \]

Where for the case of $ z^3 = -1 $, $ r = \sqrt[\leftroot{-2}\uproot{2}N]{r} = 1 \theta = pi , N = 3,$ and $ p = 1,2,3 $
So, the cube roots of -1 are 
\[ \sqrt[\leftroot{-2}\uproot{2}3]{-1} = \cos{(\frac{\pi}{3})} + i\sin{(\frac{\pi}{3})} = 1/2 + i\sin{\frac{\pi}{3}}\]
\[ \sqrt[\leftroot{-2}\uproot{2}3]{-1} = \cos{(\frac{\pi}{3}} + \frac{2\pi}{3})} + i\sin{(\frac{\pi}{3}} + \frac{2\pi}{3})} = 1\]
\[ \sqrt[\leftroot{-2}\uproot{2}3]{-1} = \cos{(\frac{\pi}{3}} + \frac{4\pi}{3})} + i\sin{(\frac{\pi}{3}} + \frac{4\pi}{3})} = 1/2 - i\sin{\frac{\pi}{3}}\]

\section*{ Problem 2 Matlab Code}
\lstinputlisting{prob2.m}

\newpage
\section*{ Problem 2 Matlab Output}
\begin{verbatim}
>> prob2

a =

  -1.0000 + 0.0000i
   0.5000 + 0.8660i
   0.5000 - 0.8660i


check1 =

    -1


check2 =

  -1.0000 + 0.0000i


check3 =

  -1.0000 - 0.0000i

>> 
\end{verbatim}






%----------------------------------------------------------------------------------------
%  Problem 4
%----------------------------------------------------------------------------------------

%%%%%%%%%%%%%%%%%%%%%%%%%%%%%%%%%%%%%%%%%%%%%%%%%%%%%%%%%%%%
\newpage
\section*{Problem 4}
\subsection*{Problem Statement (parts 1 and 2)}

Convert decimal to binary using MATLAB commands. Then convert back. Also, use MATLAB to convert 325.499 to
a 16-bit integer.

\section*{ Problem 4 Matlab Code}
\lstinputlisting{prob3.m}

\section*{ Problem 4 Matlab Output (parts 1 and 2)}
\begin{verbatim}
>> prob3

a =

1011


b =

    11


intVal =

    325


ans =

     2

>> 
\end{verbatim}

\subsection*{Problem Statement (part 3)}
Convert 0.3891 to 8-bit binary by hand.

\[ 0.3891 \]
\[ \underline{\hspace{0.8cm}2} \]
\[ (0).7782 \]
\[ \underline{\hspace{0.8cm}2} \]
\[ (1).5564 \]
\[ \underline{\hspace{0.8cm}2} \]
\[ (1).1128 \]
\[ \underline{\hspace{0.8cm}2} \]
\[ (0).2256 \]
\[ \underline{\hspace{0.8cm}2} \]
\[ (0).4512 \]
\[ \underline{\hspace{0.8cm}2} \]
\[ (0).9024 \]
\[ \underline{\hspace{0.8cm}2} \]
\[ (1).8084 \]
\[ \underline{\hspace{0.8cm}2} \]
\[ (1).6096 \]

So 0.3891 can be represented by the following 8-bit binary number 0.01100011

\subsection*{Problem 4 (part 4)}
The harmonic series diverges.



%----------------------------------------------------------------------------------------
%  Problem 5
%----------------------------------------------------------------------------------------

%%%%%%%%%%%%%%%%%%%%%%%%%%%%%%%%%%%%%%%%%%%%%%%%%%%%%%%%%%%%
\newpage
\section*{Problem 5}
\subsection*{Problem Statement (part 1)}

Derive the Taylor series for $e^{i\theta}$, $\cos{\theta}$, and $sin\theta$ and
show that the Euler equation is correct.

The general formula for the Taylor series is as follows:
\[ &f(x) \\ &= f(a) + f'(a) (x-a) + \frac{f''(a)}{2!} (x - a)^2 + \frac{f^{(3)}(a)}{3!} (x - a)^3 + \dots + \frac{f^{(n)}(a)}{n!} (x - a)^n + \cdots \end{align} \]
The Taylor series for $e^x$ can be derived as follows:
\[f^{(n)}(x)=e^x \] 
\[f^{(n)}(0)= 1 \]
for $&n=0,1,2,3...$

Using the general formula for a Taylor series and centering it at the origin (Maclaurin series), we arrive
at the $e^x$ series expansion:
\[e^x = 1 + x + \frac{x^2}{2} + \frac{x^3}{6} + \cdots = \sum_{n\geq 0} \frac{x^n}{n!}\]
Which converges for all values of x

The Taylor series for $\cos(x)$ can be derived as follows:

f^{(n)}(x)=\left\{\begin{matrix}
-\sin x &n=1,5,9,...\\ 
-\cos x &n=2,6,10,...\\ 
\sin x &n=3,7,11,...\\ 
\cos x &n=0,4,8,12,... 
\end{matrix}\right.
f^{(n)}(0)=\left\{\begin{matrix}
-1 &n=2,6,10,...\\ 
1 &n=0,4,8,12 \\ 
0 &\text{otherwise}
\end{matrix}\right.

Using the general formula for a Taylor series and centering it at the origin (Maclaurin series), we arrive
at the $\cos{x}$ series expansion:
\[\cos x = \sum_{n=0}^\infty{\frac{\cos^{(n)}(0)}{n!}x^n} \]
Which converges for all values of x

\newline
The Taylor series for $\sin(x)$ can be derived in a very similar manner to the cosine series:


\[\sin(x)=\sum\limits_{n=0}^\infty \dfrac{x^{2n+1}}{(2n+1)!}\cdot(-1)^n = x-\dfrac{x^3}{3!}+\dfrac{x^5}{5!}-\dfrac{x^7}{7!}\pm\dots\]


So, with three series defined, it's easy to show:
\[e^{ix} = 1 + x + \frac{(ix)^2}{2} + \frac{(ix)^3}{6} + \cdots = \sum_{n\geq 0} \frac{(ix)^n}{n!}\]
\[ = 1 + ix + \frac{-x^2}{2} + \frac{-ix^3}{6} + \cdots \]
\[ = (1 + ix + \frac{-x^2}{2} + \frac{-ix^3}{6} + \cdots ) + i (x - \dfrac{x^3}{3!} + \dfrac{x^5}{5!} - \dfrac{x^7}{7!} ...) \]
\[ = \cos{x} + i\sin{x} \]

\subsection*{Problem Statement (part 2)}
Starting from Euler's formula, show that the following two equations holds:
\[\cos (\theta) = \frac{e^{i\theta}+e^{-i\theta}}{2}\]
\[\sin (\theta) = \frac{e^{i\theta}-e^{-i\theta}}{2i}\]

Starting with Euler's forumla and the sum of $e^{i\theta}$
\[e^{\pm i\theta } = \cos \theta \pm i\sin \theta \]
\[e^{i\theta } + e^{-i\theta} = \cos \theta + i\sin \theta + \cos\theta - i\sin \theta  = 2\cos \theta\]
So,
\[\cos (\theta) = \frac{e^{i\theta}+e^{-i\theta}}{2}\]
A similar thing can be done to prove the $\sin \theta$ identity:
\[e^{i\theta } - e^{-i\theta} = \cos \theta + i\sin \theta - \cos\theta + i\sin \theta  = 2i\sin \theta\]
So,
\[\sin (\theta) = \frac{e^{i\theta}-e^{-i\theta}}{2i}\]
\end{document}

