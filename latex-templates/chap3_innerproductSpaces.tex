\documentclass{amsart}
\usepackage{fullpage}
\usepackage{amsmath, amsfonts, amscd, epsfig, amssymb, amsthm}

\begin{document}

%\newcounter{chapter}
%\numberwithin{section}{chapter}
\theoremstyle{definition}
\newtheorem{exercise}{Exercise}
\newcommand{\newproblem}[2]{
\setcounter{exercise}{#1}\addtocounter{exercise}{
    -1}\begin{exercise}#2\end{exercise}
    }
\newcommand{\setcontext}[2]{\setcounter{chapter}{#1}\setcounter{section}{#2}}
\newcommand{\s}[1]{\mathcal{#1}}
\newcommand{\R}{\mathbb{R}}
\newcommand{\N}{\mathbb{N}}
\newcommand{\F}{\mathbb{F}}
\newcommand{\C}{\mathbb{C}}
\renewcommand{\labelenumi}{(\alph{enumi})}


\title{Math Bootcamp Chapter 2 HW}
\author{Jeremy Bejarano}
\maketitle

%\setcontext{1}{1}

\newproblem{1}{Verify the polarization and parallelogram identities on a real vector space:
\begin{enumerate}
\item $\langle x,y \rangle = \frac{1}{4} (\| x + y \|_2^2 - \|x - y\|_2^2)$
\item $\|x\|_2^2 + \|y\|_2^2 = \frac{1}{2} (\|x + y\|_2^2 + \|x - y\|_2^2)$
\end{enumerate}
}
    \begin{proof}
    Starting with the right-hand side of part (a).
    \begin{align*}
    \frac{1}{4} (\| x + y \|_2^2 - \|x - y\|_2^2) &=  \frac{1}{4}(\langle x+y,x+y\rangle - \langle x-y,x-y \rangle) \\
    &= \frac{1}{4}(\langle x+y,x\rangle + \langle x+y,y\rangle - \langle x-y,x-y \rangle) \\
    &= \frac{1}{4}(\langle x+y,x\rangle + \langle x+y,y\rangle - (\langle x-y,x \rangle - \langle x-y,y \rangle )) \\
    &= \frac{1}{4}(\langle x,x\rangle + \langle y,x\rangle + \langle x,y\rangle + \langle y,y\rangle - \langle x-y,x \rangle + \langle x-y,y \rangle ) \\
    &= \frac{1}{4}(\langle y,x\rangle + \langle x,y\rangle + \langle y,x \rangle + \langle x,y \rangle) \\
    &= \frac{1}{4}(\langle x,y\rangle + \langle x,y\rangle + \langle x,y \rangle + \langle x,y \rangle) \\
    &= \langle x,y \rangle
    \end{align*}
    
    Now, for part (b), we again start with the right-hand side.
    
    \begin{align*}
    \frac{1}{2} (\|x + y\|_2^2 + \|x - y\|_2^2) &= \frac{1}{2} (\langle x+y,x+y \rangle + \langle x-y,x-y \rangle) \\
    &= \frac{1}{2} (\langle x+y,x \rangle + \langle x+y,y \rangle + \langle x-y,x \rangle - \langle x-y,y \rangle) \\
    &= \frac{1}{2} (\langle x,x \rangle + \langle y,x \rangle + \langle x,y \rangle + \langle y,y \rangle + \langle x,x \rangle - \langle y,x \rangle - \langle x,y \rangle + \langle y,y \rangle) \\
    &= \frac{1}{2} (\langle x,x \rangle+ \langle y,y \rangle + \langle x,x \rangle + \langle y,y \rangle) \\
    &= \frac{1}{2} ( 2\langle x,x \rangle + 2\langle y,y \rangle) \\
    &= \langle x,x \rangle + \langle y,y \rangle \\
    &= \|x\|_2^2 + \|y\|_2^2
    \end{align*}

    \end{proof}

\end{document}









