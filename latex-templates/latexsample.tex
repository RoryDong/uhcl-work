\documentclass[12pt]{article}
%http://www.google.com/url?sa=t&rct=j&q=latex+hw+assignment+with+problem+numbers+and+letters&source=web&cd=3&ved=0CEMQFjAC&url=http%3A%2F%2Fmath.bd.psu.edu%2F~mlp17%2FCourseinfo%2Flatexsample.tex&ei=yBx_UZvAAueaiAKz5oCwBw&usg=AFQjCNEdQYHynYH8KFmM3xIep8d_UmM47w&sig2=5FbLuGsHX3zuqar_WBcluA&bvm=bv.45921128,d.cGE&cad=rja

% ========================================================================
% preamble
% ========================================================================

% (stuff that you don't have to mess with or worry about right now.)

% margins, size, formatting
\oddsidemargin=.05in
\evensidemargin=.05in
\topmargin=-.5in
\textwidth=6in
\textheight=9in
\pagestyle{empty}

% packages for fancy fonts, symbols, thm/proof environments, etc
\usepackage{amsmath,amssymb,amsthm}

% The format for HW problems
% .................................................................
% usage example (pretend it is problems #14.7 with parts c and f and #14.9 with parts a, b, and c):
%
% \begin{itemize}
% \item[{\bf \large 14.7}]
% Here is where you type your answer to the problem.  If there were
% parts to the problem you would do something like the following:
%
% \begin{enumerate}
% \item[(c)]
% Here is where you type the answer to the first part.
%
% \item[(f)]
% Here is where you type the answer to the second part.
% \end{enumerate} (This ends the subproblems to #14.7)
%
% \item[{\bf \large 14.9}]
%
% \begin{itemize}
% \item
% Notice that since we are not skipping parts as in #14.7, we don't need
% to specify parts a and b because LaTeX will automatically generate them.
%
% \item
%
% \item
%
% \end{enumerate} (This ends the subproblems to #14.9)
% \end{enumerate} (This ends all of the hw problems)
% .................................................................


% ========================================================================
% body
% ========================================================================

\begin{document}

% header =================================================================
% PLEASE FILL IN THE CORRECT COURSE (e.g., MATH 311W),
% THE CORRECT ASSIGNMENT NUMBER, THE CORRECT PROBLEM TYPE
%(i.e., 10pt problems or 20 pt problems), AND YOUR NAME HERE
\centerline{\Large \bf MATH 311W Homework \#6}
\vspace{\baselineskip}
{\large \bf 10 point problems \hfill Sally Student}
\vspace{\baselineskip}

\begin{enumerate}
% problem ================================================================
\item[{\bf \large 14.1}] % notice the problem number here
This sample document provides a template for writing a \LaTeX\
document suitable for homework assignments in Dr. Previte's Math
311W class. The first few ``problems'' will explain how the
document works. Then there will be some ``problems'' that
illustrate various notations and formatting environments.  Compare
what is in the typeset version of this document to the file
\verb|latexsample.tex|.  Note in particular that anything typed
after a percent sign in the text file is treated as a comment and
is ignored by the compiler. Comments in the text file refer both
to \LaTeX\ and to hints about writing good solutions and proofs.
% subproblems ================================================================
\begin{enumerate}
\item[(b)] %notice that we need to specify the part we want when we skip a part (part a, in this case)
Notice that you don't have to do every part of a problem.  Here, we skipped part a.

\item[(e)]
Now we skip to part e.

\end{enumerate}


% problem ================================================================
\item[{\bf \large 14.4}]
All of your problem solutions should be contained in an
``enumerate'' environment.  Problems with parts will also need
``enumerate'' environments.  See the
comment in the text file for the general usage format for these
environments. Please use these environments so your homework will
be in the standard format for the class.  Also please put the
assignment number, the problem type and your name at the top of
the page as they are in this sample document.


% problem ================================================================

\item[{\bf \large 14.5}]
The very basics of \LaTeX\/ (compare the typeset document and the text file):

\begin{enumerate}
\item
Extra spaces    in           the text              file do     not    appear   in    the     typeset     document.

Except for a double carriage-return, that makes a new line.
Single
carriage
returns
don't
do
anything.


\item
Mathematical expressions are typed between dollar signs like this:
$y=x^2+1$.  To make a centered equation on its own line, use
double dollar signs, like this:
%
$$y=x^2+1.$$

The percent sign above the equation in the text file just makes it
so that extra space is not added between the centered equation and
the main paragraph above.


\item
Many \LaTeX\ commands and math symbols start with a backslash
symbol.  For example, $\sin x$ and $\{x \in \mathbb{R} : x \geq
0\}$.  Notice that the set-notation parentheses need to have
backslashes before them (while regular parentheses do not).  This
is because in \LaTeX, those squiggly parentheses often have other
uses.


\item
If you need to put something in italics you do it {\em like this}.
Or maybe you need to have something in {\bf boldface}.  Or maybe
{\bf \em both}.

\item
Notice that to have quotes appear ``correctly'' in the typeset
document you may have to type them yourself using the \verb|`| and
\verb|'| keys instead of the \verb|"| key.


%\item
%Don't forget to end each of your environments.  In other words,
%don't forget your \verb|\end{enumerate}|, or
%you will get a compiling error.  Also make sure to end dollar sign
%environments and parentheses.

\end{enumerate}

% problem ================================================================
\item[{\bf \large 15.3}]
Here is some random notation you might need:
\vspace{.5\baselineskip}
% this "vspace" above just adds some vertical space.  note that
% we can't add vertical space with carriage returns, so we have to
% add it this way instead.  here "baselineskip" is the space of a line,
% so we're skipping by half that.

$x_2$,
$x_{25}$, % note parentheses needed to get both digits in subscript
$x^2$,
$x^{25}$, % note parentheses needed to get both digits in exponent
$\pm 4$,
$x \not = 17$, % you can put "not" in front of lots of different operators
$x > 5$,
$x < 5$,
$x \geq 5$,
$x \leq 5$,
$\{ 1, 2, 3 \}$, % note curly brackets need a backslash or they are invisible
$\{ x : \sqrt{x} > 2 \}$, $\infty$.
\vspace{.5\baselineskip}

$A \subset B$,
$A \subseteq B$,
$A \not \subset B$,
$A \not \subseteq B$,
$A \setminus B$,
$A^{\rm c}$, % "rm" changes the font to "roman", i.e. non-math, font
$A \cap B$,
$A \cup B$,
$x \in A$,
$x \not \in A$,
$|A|$,
$\mathcal{P}(A)$,
$\emptyset$.
\vspace{.5\baselineskip}

$\overset{\mathcal{P}}{\equiv}$,
$\frac{5}{1+x}$,
$\displaystyle\frac{5}{1+x}$, % anything in $$ is automatically displaystyle
$\bigcap_{i=1}^n S_i$,
$\displaystyle\bigcap_{i=1}^n S_i$,
$\bigcup_{i=1}^n S_i$,
$\displaystyle\bigcup_{i=1}^n S_i$,
$\sum_{k=1}^{10} a_k$,
$\displaystyle\sum_{k=1}^{10} a_k$,
$\prod_{k=1}^{10} a_k$,
$\displaystyle\prod_{k=1}^{10} a_k$.
\vspace{.5\baselineskip}
% "displaystyle" is the default when using double dollar signs.
% so you only need to use "displaystyle" in the rare case where you
% want one of these oversized notations right in the middle of a line
% of text, which is not usually what you want.  for centered equations,
% everything will automatically be in "displaystyle".

$\mathbb{R}$, % use this ONLY to denote the real numbers
$\mathbb{Q}$, % rational numbers
$\mathbb{Z}$, % integers
$\mathbb{N}$, % natural numbers
$\clubsuit$,
$\diamondsuit$,
$\heartsuit$,
$\spadesuit$,
$\rightarrow$,
$\leftarrow$,
$\leftrightarrow$,
$\longrightarrow$,
$\longleftarrow$,
$\longleftrightarrow$,
$\Rightarrow$, % the "implies" arrow
$\Leftarrow$,
$\Leftrightarrow$, % the "if and only if" arrow
$\Longrightarrow$, % longer "implies" arrow
$\Longleftarrow$,
$\Longleftrightarrow$, % longer "if and only if" arrow
$\mapsto$,
$\longmapsto$.
\vspace{.5\baselineskip}

$\mathcal{P}$, % "mathcal" is a fancy font that can be applied to any letter.
$\mathcal{S}$,
$\mathcal{F}$,
$\forall$,
$\exists$,
$\lor$, % think "logical or"
$\land$, % think "logical and"
$\neg$, % \lnot also works.  use this for logical negation (\sim looks funny)
$\sim$, % use this for equivalence relations, it's made to be a binary operation
$\approx$,
$\equiv$,
$\times$, % for cartesian products
$\ast$,
$\star$,
$a | b$, % use this for "divides"
$|x|$, % use this for absolute value
$\|x\|$,
$\lceil x \rceil$,
$\lfloor x \rfloor$,
$\{x \in \mathbb{Z} \mid x \mbox{ is prime} \}$. % note use of "mbox"
% the "mbox" is needed so that we can have non-math type inside of the
% math environment.  without the "mbox" the words would be in math/italics,
% and all smushed together with no spaces between words.  notice also
% the space before the word "is".
\vspace{.5\baselineskip}

$\gcd$, % in math mode, just "gcd" would be in italics, but "\gcd" is not
${\rm lcm}$, % there isn't a command for "lcm" in tex so we just roman it
$n \choose k$,
$n+1 \choose k$, % no brackets needed, the n+1 is all assumed to be on top
$a = {n+1 \choose k}$, % we need brackets or else "a=" would be in the choose
$\prec$, % think "precedes"
$\preceq$,
$\succ$, % think "succeeds"
$\succeq$,
$f \colon [0,\infty) \rightarrow \mathbb{R}$,
%\colon above looks better than using : in this application
$f \circ g$ % composition
\{, % these next few symbols mean particular things to latex
\}, % so to get them to appear in your document you precede them with backslash
\$, % notice that these are NOT in math mode
\%,
\&,
\_,
\#.


% problem ================================================================
\item[{\bf \large 15.9}] Suppose 43 students take algebra,
32 take Spanish, 7 take both. \vspace{.5\baselineskip}
% i like to put a little space between the initial problem information
% and the solution, just to make it look nice.  note this isn't necessary
% if the solution has parts (see above) or is a proof (see below)

The number taking algebra or Spanish (or both, of course) is:
%
$$43 + 32 - 7 = 68.$$

(We have to subtract 7 because those students are counted twice.)

% for this problem, notice that i didn't just answer "68".
% there is work!  and justification!  and some info at the beginning!



% problem ================================================================
\item[{\bf \large 15.10}]
You don't have to do your truth tables in \LaTeX; you may write
them in by hand if you like.  But in case you are interested, this
is how to do it:

\begin{center}
\begin{tabular}{|c|c||c|c|c|c|}
\hline
$x$ & $y$ & $x \land y$ & $\neg(x \land y)$ & $\neg y$ & $\neg(x \land y) \land \neg y$ \\
\hline
T & T & T & F & F & F \\
T & F & F & T & T & T \\
F & T & F & T & F & F \\
F & F & F & T & T & T \\
\hline
\end{tabular}
\end{center}

Since the truth-values for $\neg y$ and $\neg(x \land y) \land
\neg y$ are the same for all possible truth-values of $x$ and $y$,
the two statements are logically equivalent.


% problem ================================================================
\item[{\bf \large 14.2}]
Prove that if $x$ and $y$ are both odd, then $x\equiv y \mod 2$.
\begin{proof} % notice there is a built-in proof environment - use it!
Let $x$ and $y$ be odd integers. %restating the hypothesis
Then by definition of odd, there exist integers $m$ and $n$ such that $x = 2m+1$ and $y= 2n+1$. %unravelling the def of the term in the hypothesis
Observe %here comes the link
%
\begin{align*} % the "%" on the line above just prevents added linespace
x-y  &= (2m+1)- (2n+1) % "&=" gives an aligned equals.  the "&" is like "tab"
     &\mbox{(since $x=2m+1$ and $y=2n+1$)} \\ % this is how you might provide a reason
  &= 2m+1 -2n-1 \\
  &= 2m -2n \\
  &= 2(m-n).  % notice we are still using punctuation, even here!
     &\mbox{(factor out a $2$)}
\end{align*}
% the "*" in the align environment just makes it so the equations are not
% numbered.  in this example the reasons/justifications given for the
% steps aren't really mathematically needed - they are pretty obvious - but
% i put them here so you could see how to include them when necessary.

So there exists $k \in \mathbb{Z}$, namely $k=m-n$, such that $x-y=2k$.
Therefore, $2|(x-y)$.  % notice that i am unravelling the definition of the conclusion here
Hence, $x\equiv y \mod 2$.

\end{proof} % ending the proof environment automatically adds the "box"


\vspace{2\baselineskip} %i put in some extra space may be helpful here to make seeing the two parts easier
Prove that if $x$ and $y$ are both even, then $x\equiv y \mod 2$.
\begin{proof}
Then by definition of even, there exist integers $m$ and $n$ such that $x = 2m$ and $y= 2n$. %unravelling the def of the term in the hypothesis
Observe %here comes the link
%
\begin{align*}
x-y  &= 2m- 2n
     &\mbox{(since $x=2m$ and $y=2n$)} \\
  &= 2(m-n).
     &\mbox{(factor out a $2$)}
\end{align*}

So there exists $k \in \mathbb{Z}$, namely $k=m-n$, such that $x-y=2k$.
Therefore, $2|(x-y)$.  % notice that i am unravelling the definition of the conclusion here
Hence, $x\equiv y \mod 2$.

\end{proof} %


\end{enumerate} % don't forget to end the enumerate environment after all that!


\end{document} % every document must end with this.
