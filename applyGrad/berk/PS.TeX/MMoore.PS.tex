\documentclass{article}
\usepackage[
  letterpaper,
  margin=1in,
  headsep=4pt, % separation between header rule and text
]{geometry}
\usepackage{xcolor}
\usepackage{fancyhdr}

\pagestyle{fancy}
\fancyhf{}
\fancyhead[C]{%
  \footnotesize\sffamily
  \yourname\quad
  \textcolor{blue}{\youremail}}

\newcommand{\soptitle}{Personal History Statement}
\newcommand{\yourname}{Michael Moore}
\newcommand{\youremail}{michael.moore@nasa.gov}

\newcommand{\statement}[1]{\par\medskip
  \underline{\textcolor{blue}{\textbf{#1:}}}\space
}

\usepackage[
  colorlinks,
  breaklinks,
  pdftitle={\yourname - \soptitle},
  pdfauthor={\yourname},
  unicode
]{hyperref}

\begin{document}

\begin{center}\LARGE\soptitle\\
\large EECS MS/PhD applicant for Fall---2015
\end{center}

\hrule
\vspace{1pt}
\hrule height 1pt

\bigskip

I still remember the strange feelings of inadequacy after receiving my B.S. in engineering.
Surely, by then, I should have felt like a confident young engineer. I had the credentials,
yet I still remember showing up to my new job with the fear of being discovered as
some sort of fraud. Thankfully after six months or so, these feelings began to subside. I
had the realization that my education was just beginning. Since then, I have been happily
along for the ride. My time in industry has given me some remarkable opportunities to
naturally develop into the engineer I naively expected myself to be following graduation.
Now as I apply for graduate school, I expect, perhaps naively once again, that I 
will truly be ready upon arrival. I intend to show that my time away from school has
provided me with invaluable experience in giving high stakes presentations, conducting
laboratory research, and acting as a leader and mentor for other young engineers.
All of these experiences should facilitate my success in Berkeley's EECS graduate program.

Three and a half years after first arriving at my job at Johnson Space Center (JSC),
I was introduced to a NASA graduate intern named Zu Qun Li. I had been tasked by boss to help
get Zu Qun up to speed so that he could help with my work on a software model of a
candidate life support system design for a deep space habitat for astronauts.
Zu Qun was expected to build a software model of a Sabatier Reactor in two months time.
This reactor is a component within the life support system that recycles
carbon dioxide and hydrogen into water and methane. As a graduate student at Penn State, Zu Qun
was familiar with the underlying numerical techniques required to build such a model, but
he was less familiar with some of the software design patterns that I was using to build
the software architecture for the entire life support system. I showed him 
how to approach the problem in such a way that would lend itself well to this architecture.
Additionally, I created materials for him to use to learn our set of tools and processes. Most importantly,
I helped introduce him to some of the experts he would need to work with, and I peer reviewed
his papers and presentations. Zu Qun finished the Sabatier Reactor model on time, and
his exit presentation to NASA was well received. I am happy to say that he will be
joining our team as a full time employee in January 2015. Mentorship, collaboration and
peer-review are important aspects of graduate school that I believe this experience has
helped prepare me for.

Another important asset in graduate school is the abilitiy to promote and communicate your
work to stakeholders. This is an area where I know I have an edge over graduate students
that plan to attend immediately following their graduation. A significant aspect of my
job is demonstrating my work to the people at NASA headquarters who directly fund our
efforts. I give several of these presentations each year. I'm one of the few on the
team who the bosses trust to present directly to the headquarter's management. At first,
this was a scary experience for me. Now I accept it as just another important part of my job.
This experience should serve me well in graduate school whenever I need to get people interested
in my research, demonstrate progess to stakeholders, and represent the department at conferences.

Outside of work, I have continued to prepare myself for a graduate degree in EECS. After
my work day, I take graduate night classes at a local university in topics related to
computer engineering and electronics. My goal in doing this is not for a promised
salary increase, but rather to fill in some gaps in my formal electronics education.
An undergraduate Aerospace Engineering degree offers some broad training in electronics,
instrumentation, and control systems. However, I have found that taking extra night classes
can really go a long way. Of all the things I have learned from classes in signals and systems, 
computer architecture, and wireless communications, it is probably the time management
lessons that will stick with me the longest. Working full time and taking classes is
no easy task. I look forward to the day where I can focus primarily on graduate
research. I know there's more to graduate school than taking classes, and
I have done well to prepare myself to manage the time committments to extra-curriculur
activities expected from a Berkeley graduate student.

One such activity that I still find time to enjoy is my involvement
in hobbyist robotics projects. Myself and two NASA friends have formed a local robotics club
in which we share access to lab equipment that we jointly own. 3D printers, signal anaylzers, 
soldering equipment, and microcontrollers are some common examples of often used equipment.
Projects tend to involve building various gadgets mostly for fun. Some examples include a 
tachometer for a friend's car, a wireless IMU sensor platform, and various remote controlled
vehicles. Probably the most impressive project is a flight controller for a custom built quad
copter. The most enjoyable memory I have from that project is spending all night in the garage
working on a problem related to sensor fusion. Before we could achieve good control, it was
critical to get reliable measurements of the vehicle's orientation. I worked late into the
night attempting to apply a quaternion based gradient descent algorithm that I had read about in
a research paper on adaptive filter implementations for IMUs. After a few adjustments to the
algorithm, and some debugging with the signal analyzers, I was finally able to get 
it to work around 7 AM the following morning. Seeing clean roll, pitch, and yaw measurements 
coming in for the first time was a great feeling.
After all my time spent in that garage, I feel confident that I could be a great asset in a legitimate research laboratory.  

I feel that I have matured a great deal as an engineer after departing U.T. I confidently 
welcome the unknown challenges that lie ahead in my career. Since graduation, I have filled my time
with productive activities for an aspiring graduate student. My time in industry has provided me
with some unique opportunities that should serve me well as I transition back to academia.
I am eager for the opportunity to attend Berkeley, and start the new chapter in my engineering development. 

\end{document}