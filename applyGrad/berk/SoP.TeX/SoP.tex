\documentclass{article}
\usepackage[
  letterpaper,
  margin=1in,
  headsep=4pt, % separation between header rule and text
]{geometry}
\usepackage{xcolor}
\usepackage{fancyhdr}

\pagestyle{fancy}
\fancyhf{}
\fancyhead[C]{%
  \footnotesize\sffamily
  \yourname\quad
  web: \textcolor{blue}{\itshape\yourweb}\quad
  \textcolor{blue}{\youremail}}
\fancyfoot[C]{Page \thepage\ of \pageref{LastPage}}

\newcommand{\soptitle}{Statement of Purpose}
\newcommand{\yourname}{Michael Moore}
\newcommand{\youremail}{michael.moore@nasa.gov}
\newcommand{\yourweb}{http://www.hardworkin-man.com/}

\newcommand{\statement}[1]{\par\medskip
  \underline{\textcolor{blue}{\textbf{#1:}}}\space
}

\usepackage[
  colorlinks,
  breaklinks,
  pdftitle={\yourname - \soptitle},
  pdfauthor={\yourname},
  unicode
]{hyperref}

\begin{document}

\begin{center}\LARGE\soptitle\\
\large of \yourname\ (EECS MS/PhD applicant for Fall---2015)
\end{center}

\hrule
\vspace{1pt}
\hrule height 1pt

\bigskip

It is my hope that a graduate education within UC Berkeley's EECS department will push me
to the cutting edge of the fields of robotics, artifical intelligence, and control
systems. I must admit that I didn't start out with this goal. I received my B.S. in Aerospace Engineering at the University of Texas at Austin in 2010. I then spent the past four years working as 
an engineer at NASA Johnson Space Center within the Robotics, Software, and Simulation branch. I love my job
and it represents the fulfillment of a life long dream. However, as I have advanced in
my career, I have come to realize that this dream job comes with a serious set of responsibilities.
The field in which I work still has many open and unsolved problems. I find that most exciting, and
I want to do everything within my capacity to help develop and ultimately succeed within this
technically demanding field. In following one dream, I have discovered that I have yet another. I believe a graduate education at UC Berkely that focuses on robotics, artifical intelligence, and control systems
will help me realize my ambition to contribute to, and succeed within the emerging field of robotics.

Unlike many other undergraduate Aerospace Engineers, I often sought out projects that involved
large amounts of mathematical modeling and simulation. I commonly fought the urge
to use Matlab to solve all of my problems in numerical simulation. I favored the experience of learning
C++ and Python along with a host of open source scientific libraries in order to do my work. This 
was commonly at odds with the standard Aerospace curriculum, and it often required me to do more work.
That was alright with me, however, because programming and learning in this way has been a hobby of mine
since grade school. Thankfully, this way of doing things quickly paid off. As a junior, I was able to get
a job with my Systems Engineering professor doing undergraduate research at UT's Center for Space Research (CSR).
In this role, I was able to apply my knowledge of Unix, C, and Python programming to help less
programmatically inclined Aerospace graduate students post process scientific instrument data being downlinked
from the pair of GRACE satellites which UT CSR helps NASA to operate. In parallel with this effort,
I was also spending late nights writing software to develop a model to help in my group's senior design project.
I was able to build a math model and then simulate some of the thermodynamics involved in a hypothetical
robotic space mission entering the atmosphere of Venus. At the end of our project, I presented my model to a
large group of people in industry. My software model was of particular interest to NASA's Jet Propulsion
Laboratory who was actively conducting design trade studies as a part of considering a future robotic 
mission to Venus. Based on my presentation, they flew me out to Pasadena for a job interview. In the end,
I took a job in Houston at L3 Stratis, who specialized in modeling and simulation work for NASA's Johnson
Space Center. (JSC Trick Initiative? I probably don't want to leave it out completely.)


I have now been working for L3 Stratis on-site at NASA Johnson Space Center for the past four years. 
I started out doing modeling and simulation work for NASA's Orion vehicle. I worked with my mentor
Jason Harvey to help design and create a nodal network solver that went on to become the underlying solver architecture for the majority of the vehicle simulation's electrical, thermal, and life support systems.
The solver relies on nodal analysis techniques used within the electrical engineering discipline. My mentor and I
then used object-oriented programming techniques (most notably polymorphism), and the hydro-electric
analogy to build a generic solver that could be applied to electrical, thermal, and fluidic systems.
After successly demonstrating early working versions of this nodal network solver, I was moved
to a small team who was tasked in further demonstrating some of the software architecural principles
 that I had helped to develop.
Before committing to a large investment in updating their existing Space Station simulator, NASA wanted to verify that the software architecture being proposed could scale to a fully integrated vehicle simulation. To meet these ends, I was tasked with helping to demonstrate the architecture on a simulation of the Japanese
HTV (H-II Transfer Vehicle).
This project ended up being a large success, and demonstrated that our solver architecture
could adequately simulate a spacecraft's electrical, thermal, and life support system. After the 
HTV project, I was moved into the simulation team within JSC's Advanced Exploration Systems.
This project was intended to evaluate the use of our software to aid in the design of future
NASA vehicles. Specifically, proposed vehicles for deep space and Mars missions. This is
the team I am currently on. The project manager for our team, Dr. Edwin Crues, is a skilled and ambitious leader with the goal of impacting the design of NASA's future Mars mission early on in the design life cycle. 
Our team builds software models of the current state of the vehicle design. NASA uses our software
models to conduct trade studies to evaluate the design in the context of a given mission architecture.
In particular, I am the modeling and simulation lead for the thermal, electrical, and life support
subsystems. This project has been great experience for me, and it is by far the best team I have worked
with at NASA. My work, along with the rest of our teams work, has been demonstrated and well received 
at the highest levels of NASA, including Dr. Jim Green, NASA's Planetary Science Division Director, and 
the NASA administrator himself, Charlie Bolden.



    - Describe why and wtf you are doing at UHCL. Free paid for education by my company. Remember how I said 
    I love this stuff. This is how I spend my free time. Night classes at the closest university. Taking advantage of my companies continuing education benefit. It has never been about the master's degree. I just needed to learn some new things in order to build my hobbyist robotics projects.

What I want to study in grad school.
I'm currently taking graduate courses in digital signal processing and control systems. This is to take advantage
of my companies continuing education benefit, and also to prepare myself for serious graduate study and
research. As an Aerospace Engineer, I much appreciated more in depth courses in digital circuits, comp arch,
wireless communication, etc.
Outside of full time work and night school, I also currently work with a friend and graduate student at Texas AM 
on a research project. We have built a custom quadcopter, including the flight control system. We currently use it as a test bed for control systems research. This has been a truly ground up effort, and I had to attach supporting
design documentation and sucessful ground and flight test videos in order to do it justice.

Being more specific, I have particular interest and experience in wireless sensor systems and communication networks as input to high level controllers that control
    based on machine learning concepts.
    - Describe your IMU swarmlet idea.
    - List a few other ideas. Specific professors that I believe I could work well with include:

       - Claire Tomlin
       - Ed Whateer
       - That other dust sensor dude


Part 5: Conclusion
• End your statement in a positive and confident manner with a readiness for the challenges of graduate study.

Conclusion
    - I truly don't want this thing to be super long and go unread, but I have only talked about a few of my
    major projects. I have so many other things going on in parallel. I do indie game development. I contribute to open source projects. I enjoy web-app development and cloud based computing. I have written back-ends for web
    applications. UCB is leading the way with their TerraSwarm research lab. They are trying to show Silicon
    Valley and the rest of the country that it can't all be about software. The cloud ( I really hate that term )
    is not just useful as a distributor of software services. This is so incredibly short sighted it is silly. The 
    future of these technologies is not about storing photos and swapping likes. It is the foundation of  

UC Berkeley is an ideal school for me. 


extra stuff.

 Example of WRS. Co author and submitting a paper to ICES conference. I'm supplying all of the data/plots/analysis work. The other co-author is a design expert and he's providing the context and potential application of high fidelity simulations to aid in NASA's design process.

  - I know how to track down funding. In my job I often help present to higher up NASA HQ people in order to get funding. I wasn't always comfortable with public speaking, but my job has forced me to get good at it. I know the value of this skill, and it's a rarity among graduate engineering students. I can present at conferences. I can clearly describe the work that's being done and why it's important. I can help attract new grants/fudning oppurtunities. I have done it at my job at NASA. 

  - I have practical engineering experience. I know how to write software on a team. I know how to architect software and the importance of testing. I have solved several critical integration problems at work. I know how to work 
  late into the night and get shit done. I know how to manage my time, and not waste effort in unfruitful areas.


\end{document}