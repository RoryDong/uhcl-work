\documentclass{article}
\usepackage[
  letterpaper,
  margin=1in,
  headsep=4pt, % separation between header rule and text
]{geometry}
\usepackage{xcolor}
\usepackage{fancyhdr}

\pagestyle{fancy}
\fancyhf{}
\fancyhead[C]{%
  \footnotesize\sffamily
  \yourname\quad
  web: \textcolor{blue}{\itshape\yourweb}\quad
  \textcolor{blue}{\youremail}}
\fancyfoot[C]{Page \thepage\ of \pageref{LastPage}}

\newcommand{\soptitle}{Statement of Purpose}
\newcommand{\yourname}{Michael Moore}
\newcommand{\youremail}{michael.moore@nasa.gov}
\newcommand{\yourweb}{http://www.hardworkin-man.com/}

\newcommand{\statement}[1]{\par\medskip
  \underline{\textcolor{blue}{\textbf{#1:}}}\space
}

\usepackage[
  colorlinks,
  breaklinks,
  pdftitle={\yourname - \soptitle},
  pdfauthor={\yourname},
  unicode
]{hyperref}

\begin{document}

\begin{center}\LARGE\soptitle\\
\large of \yourname\ (EECS MS/PhD applicant for Fall---2015)
\end{center}

\hrule
\vspace{1pt}
\hrule height 1pt

\bigskip

Introduce myself and quick overview of my background and general interests.
Aerospace Engineer, particularly interested and skilled in computers. 4 years of work exp.
Engineering is not just a job to me, it's a lifestyle. I have always been 
involved in engineering hobby projects. Some of which I will highlight in this 
statement of purpose.
I understand that continuing education is critical in order for me to get to where I want to 
be in the engineering field. I want the oppurtunity to dedicate my attention to formally learning advanced
engineering concepts and understanding the state of the art in the field of embedded systems and control.
UC Berkeley is the best fit for my educational goals.


My undergraduate degree is in aerospace engineering. I have a strong background in modelling 
system dynamics and implementing control schemes.
Unlike other aerospace engineers, I often would seek out projects that involved large amounts of 
programming, and robotics and control.
Joined the JSC Trick initiative.
Joined the CSR. Learned Linux and C. 
My senior design project. I wrote code to model the thermodynamics of a robotic mission entering 
Venus' atmosphere. We presented to a large group of people in industry. My software model was of 
particular interest to JPL. Based on my presentation, they flew me out to Pasadena to interview. In the end,
I took a job in Houston at L3 Stratis, who specialized in modeling and simulation work for JSC.


Have been working at L3 for four years. Started out doing modeling and simulation work
for NASA's Orion vehicle. I worked with my mentor Jason Harvey to help design and create
a nodal network solver that went on to become the underlying solver architecture for the
majority of the vehicle simulation's electrical, thermal, and life support systems.
After successly demonstrating early working versions of this nodal network solver, I was moved
to a small team who was tasked in demonstrating some of the software architecural principles
for the Orion simulation that NASA was beginning to invest in.
Before committing to a large investment, NASA wanted to verify that the architecture being proposed would
scale to a fully integrated Orion vehicle simulation. To meet these ends, I was moved to a small
team that was tasked with demonstrating the software architecture on a simulation of the Japanese
HTV. The simulation was to be used to train flight controller's that were training for the 
Visiting Vehicle Officer position within NASA JSC's well known Mission Control Center.
This project ended up being a large success, and demonstrated that our solver architecture
could adequately simulate a spacecraft's electrical, thermal, and life support systems.
NASA went on to use our software architecture for simulations of not only the Orion vehicle, but
also the International Space Station. These simulations were developed for the purpose of training
NASA flight controllers, but NASA understands the value of quality software models. After the 
HTV project, I was moved into the simulation team within JSC's Advanced Exploration Systems.
This project was intended to evaluate the use of our software to aid in the design of future
NASA vehicles. Specifically proposed vehicles for deep space and Mars missions. This is
the team I am currently on.
The project manager for our team, Dr. Edwin Crues, is a skilled and ambitious leader with 
the goal of impacting the design of NASA's future Mars mission early on in the design life cycle. 
Our team builds software models of the current state of the vehicle design. NASA uses our software
models to conduct trade studies to evaluate the design in the context of a given mission architecture.
In particular, I am the modeling and simulation lead for the thermal, electrical, and life support
subsystems. This project has been great experience for me, and it is by far the best team I have worked
with at NASA. My work, along with the rest of our teams work, has been demonstrated and well received 
at the highest levels of NASA, including Dr. Jim Green, NASA's Planetary Science Division Director, and 
the NASA administrator himself, Charlie Bolden.



    - Describe why and wtf you are doing at UHCL. Free paid for education by my company. Remember how I said 
    I love this stuff. This is how I spend my free time. Night classes at the closest university. Taking advantage of my companies continuing education benefit. It has never been about the master's degree. I just needed to learn some new things in order to build my hobbyist robotics projects.

What I want to study in grad school.
I'm currently taking graduate courses in digital signal processing and control systems. This is to take advantage
of my companies continuing education benefit, and also to prepare myself for serious graduate study and
research. As an Aerospace Engineer, I much appreciated more in depth courses in digital circuits, comp arch,
wireless communication, etc.
Outside of full time work and night school, I also currently work with a friend and graduate student at Texas AM 
on a research project. We have built a custom quadcopter, including the flight control system. We currently use it as a test bed for control systems research. This has been a truly ground up effort, and I had to attach supporting
design documentation and sucessful ground and flight test videos in order to do it justice.

Being more specific, I have particular interest and experience in wireless sensor systems and communication networks as input to high level controllers that control
    based on machine learning concepts.
    - Describe your IMU swarmlet idea.
    - List a few other ideas. Specific professors that I believe I could work well with include:

       - Claire Tomlin
       - Ed Whateer
       - That other dust sensor dude


Part 5: Conclusion
• End your statement in a positive and confident manner with a readiness for the challenges of graduate study.

Conclusion
    - I truly don't want this thing to be super long and go unread, but I have only talked about a few of my
    major projects. I have so many other things going on in parallel. I do indie game development. I contribute to open source projects. I enjoy web-app development and cloud based computing. I have written back-ends for web
    applications. UCB is leading the way with their TerraSwarm research lab. They are trying to show Silicon
    Valley and the rest of the country that it can't all be about software. The cloud ( I really hate that term )
    is not just useful as a distributor of software services. This is so incredibly short sighted it is silly. The 
    future of these technologies is not about storing photos and swapping likes. It is the foundation of  

UC Berkeley is an ideal school for me. 


extra stuff.

 Example of WRS. Co author and submitting a paper to ICES conference. I'm supplying all of the data/plots/analysis work. The other co-author is a design expert and he's providing the context and potential application of high fidelity simulations to aid in NASA's design process.

  - I know how to track down funding. In my job I often help present to higher up NASA HQ people in order to get funding. I wasn't always comfortable with public speaking, but my job has forced me to get good at it. I know the value of this skill, and it's a rarity among graduate engineering students. I can present at conferences. I can clearly describe the work that's being done and why it's important. I can help attract new grants/fudning oppurtunities. I have done it at my job at NASA. 

  - I have practical engineering experience. I know how to write software on a team. I know how to architect software and the importance of testing. I have solved several critical integration problems at work. I know how to work 
  late into the night and get shit done. I know how to manage my time, and not waste effort in unfruitful areas.


\end{document}