\documentclass{article}
\usepackage[
  letterpaper,
  margin=1in,
  headsep=4pt, % separation between header rule and text
]{geometry}
\usepackage{xcolor}
\usepackage{fancyhdr}

\pagestyle{fancy}
\fancyhf{}
\fancyhead[C]{%
  \footnotesize\sffamily
  \yourname\quad
  \textcolor{blue}{\youremail}}

\newcommand{\soptitle}{Statement of Purpose}
\newcommand{\yourname}{Michael Moore}
\newcommand{\youremail}{michael.moore@nasa.gov}

\newcommand{\statement}[1]{\par\medskip
  \underline{\textcolor{blue}{\textbf{#1:}}}\space
}

\usepackage[
  colorlinks,
  breaklinks,
  pdftitle={\yourname - \soptitle},
  pdfauthor={\yourname},
  unicode
]{hyperref}

\begin{document}

\begin{center}\LARGE\soptitle\\
\large EECS MS/PhD applicant for Fall---2015
\end{center}

\hrule
\vspace{1pt}
\hrule height 1pt

\bigskip

Four years ago, during a break from a job interview at the Jet Propulsion Laboratory, I watched the Mars Curiosity Rover go through a meticulous inspection prior to departing for Kennedy Space Center for launch. It was the most amazing robot I had ever seen, and I knew from that point on that I had to be involved within the field of robotics. Since that visit, I have received my B.S. in Aerospace Engineering from U.T. Austin. Following that, I spent the past four years working as an engineer at NASA Johnson Space Center (JSC) within the Robotics, Software, and Simulation branch. I love my job and it represents the fulfillment of a childhood dream. However, as I have advanced in my career, that dream has given way to reality. My time at NASA has forced me to confront the immense technical obstacles that must be overcome to build complex robotic systems like the Curiosity rover. My job has inspired me to continue my education so that I can be a part of the design of future intelligent robotic systems. UC Berkeley is known around the world to be a leader in providing such an education. For these reasons, I am applying to Berkeley's EECS graduate school in order to research design, modeling, and analysis as it applies to intelligent robotic systems.


Unlike many other undergraduate Aerospace Engineers, I often actively sought out projects that involved
large amounts of mathematical modeling and simulation. At first, I used my developing skills in this area to support my extra-curricular robotics hobbies. Pretty soon, however, I was able to find a way to get paid for these same skills. In my third year, I was able to get a job with my Systems Engineering professor doing undergraduate research at U.T.'s Center for Space Research (CSR). In this role, I was able to apply my knowledge of Unix, C, and Python programming to help post process scientific instrument data being downlinked
from the pair of GRACE satellites which U.T. helps NASA to operate. In parallel with this effort,
I was also spending late nights writing software to develop a model to help in my group's senior design project.
I was able to build a math model and then simulate some of the thermodynamics involved in a hypothetical
robotic space mission entering the atmosphere of Venus. At the end of our project, I presented my model to a
large group of people in industry. My software model was of particular interest to NASA's Jet Propulsion
Laboratory who was actively conducting design trade studies as a part of considering a future robotic 
mission to Venus. Based on my presentation, they flew me out to Pasadena for a job interview. In the end,
I took a job in Houston at L3 Communications to do modeling and simulation work for JSC. 

I started out at JSC helping to build a simulator for NASA's Orion vehicle. My mentor and I built a nodal network solver that became the underlying solver architecture for the Orion simulation's electrical, thermal, and life support systems.
The solver relies on nodal analysis techniques to simulate these systems in real-time.
After successfully demonstrating early working versions of the solver, I was moved
to a small simulation team within JSC's Advanced Exploration Systems.
This project uses our software to aid in the design of future
NASA vehicles. Specifically, proposed vehicles for deep space and Mars missions. 
Our team builds software models of the current state of the vehicle design. NASA uses our
models to conduct trade studies to evaluate the design in the context of a given mission architecture.
In particular, I am the modeling and simulation lead for the thermal, electrical, and life support
subsystems. My work, along with the rest of our teams work, has been demonstrated and well received 
at the highest levels of NASA, including Dr. Jim Green, NASA's Planetary Science Division Director, and 
the NASA administrator himself, Charlie Bolden.

Berkeley's Design, Modeling, and Analysis research area is a natural fit for me. Most of my academic and professional work up to this point has involved building computational models of a variety of dynamical systems. This has given me a good basis of practical knowledge in this field. For me, the subject extends beyond academic and professional boundaries. I also employ it within my own robotics hobbies. I have worked with another colleague outside of NASA on our own UAV hobby project. We recently completed a successful flight test of our custom built quad copter. Over the course of about a year and a half, we developed a flight control system from scratch that runs on a TI CMS320 F2808 microcontroller. The design of the control system was enabled by a dynamic, real-time, 3D model and simulation that I created for the project. Over 90\% of the software for the robot is our own, including the motor drivers, wireless communication, sensors, and control system aspects. Plus, the system actually flies (refer to the video link included with my resume). I use the project as a control system test bed and Cyber-Physical System platform. As exciting as this project is for me, it currently remains a garage based hobby. I want to reach the frontiers of dynamical systems modeling and controls research. Berkeley is at that forefront, and I want to be a part of it.

\end{document}